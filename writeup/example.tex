\documentclass[a4paper]{article}\usepackage[]{graphicx}\usepackage[]{color}
%% maxwidth is the original width if it is less than linewidth
%% otherwise use linewidth (to make sure the graphics do not exceed the margin)
\makeatletter
\def\maxwidth{ %
  \ifdim\Gin@nat@width>\linewidth
    \linewidth
  \else
    \Gin@nat@width
  \fi
}
\makeatother

\definecolor{fgcolor}{rgb}{0.345, 0.345, 0.345}
\newcommand{\hlnum}[1]{\textcolor[rgb]{0.686,0.059,0.569}{#1}}%
\newcommand{\hlstr}[1]{\textcolor[rgb]{0.192,0.494,0.8}{#1}}%
\newcommand{\hlcom}[1]{\textcolor[rgb]{0.678,0.584,0.686}{\textit{#1}}}%
\newcommand{\hlopt}[1]{\textcolor[rgb]{0,0,0}{#1}}%
\newcommand{\hlstd}[1]{\textcolor[rgb]{0.345,0.345,0.345}{#1}}%
\newcommand{\hlkwa}[1]{\textcolor[rgb]{0.161,0.373,0.58}{\textbf{#1}}}%
\newcommand{\hlkwb}[1]{\textcolor[rgb]{0.69,0.353,0.396}{#1}}%
\newcommand{\hlkwc}[1]{\textcolor[rgb]{0.333,0.667,0.333}{#1}}%
\newcommand{\hlkwd}[1]{\textcolor[rgb]{0.737,0.353,0.396}{\textbf{#1}}}%
\let\hlipl\hlkwb

\usepackage{framed}
\makeatletter
\newenvironment{kframe}{%
 \def\at@end@of@kframe{}%
 \ifinner\ifhmode%
  \def\at@end@of@kframe{\end{minipage}}%
  \begin{minipage}{\columnwidth}%
 \fi\fi%
 \def\FrameCommand##1{\hskip\@totalleftmargin \hskip-\fboxsep
 \colorbox{shadecolor}{##1}\hskip-\fboxsep
     % There is no \\@totalrightmargin, so:
     \hskip-\linewidth \hskip-\@totalleftmargin \hskip\columnwidth}%
 \MakeFramed {\advance\hsize-\width
   \@totalleftmargin\z@ \linewidth\hsize
   \@setminipage}}%
 {\par\unskip\endMakeFramed%
 \at@end@of@kframe}
\makeatother

\definecolor{shadecolor}{rgb}{.97, .97, .97}
\definecolor{messagecolor}{rgb}{0, 0, 0}
\definecolor{warningcolor}{rgb}{1, 0, 1}
\definecolor{errorcolor}{rgb}{1, 0, 0}
\newenvironment{knitrout}{}{} % an empty environment to be redefined in TeX

\usepackage{alltt}

\usepackage[english]{babel}
\usepackage[utf8]{inputenc}
\usepackage{amsmath}
\usepackage{graphicx}
\usepackage[colorinlistoftodos]{todonotes}

% \title{Bayesian Model for Vehicle Crashes on Two-Lane Primary Roads in Iowa}
% 
% \author{Daniel Ries \\ Michael D. Pawlovich \\ Alicia Carriquiry \\ Zachary Hans}
% 
% \date{\today}
\IfFileExists{upquote.sty}{\usepackage{upquote}}{}
\begin{document}




\section{Examples}
\subsection{Low Crash}


  
Suppose a one mile asphalt principal arterial road in a rural area of district 1 that has a volume of 403 was built and we are interest in the expected number of crashes.

\begin{align*}
\lambda_{new} &= e^{x_{new}'\beta} = exp((1,0,0,0,0,0,log(403),0,0,0) \begin{bmatrix} -7.302\\-.118\\-.058\\-.054\\.209\\.26\\.971\\.066\\.443\\-.005 \end{bmatrix}) \\
            &= 0.22855
\end{align*}

That means we expect 0.22855 crashes (per mile) on this segment in a year. A 95\% credible interval for the expected number of crashes is (0.2088,0.2497).


\subsection{High Crash}


Suppose a one mile asphalt collector road in an urban area of district 6 that has a volume of 7000 was built and we are interest in the expected number of crashes.

\begin{align*}
\lambda_{new} &= e^{x_{new}'\beta} = exp((1,0,0,0,0,1,log(7000),0,1,0) \begin{bmatrix} -7.302\\-.118\\-.058\\-.054\\.209\\.26\\.971\\.066\\.443\\-.005 \end{bmatrix}) \\
            &= 7.3791
\end{align*}

That means we expect 7.3791 crashes (per mile) on this segment in a year. A 95\% credible interval for the expected number of crashes is (6.4954,8.381).

% 
\end{document}
